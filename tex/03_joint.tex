\documentclass[10pt]{beamer}
%\usepackage{xspace}
\usepackage{amsmath,amssymb}
\usepackage{graphicx}
%\usepackage{svg}
%\usepackage{pgfpages}
%\pgfpagesuselayout{4 on 1}[a4paper,border shrink=5mm,landscape]
%\usepackage{psfrag}
%\usepackage[usenames,dvipsnames]{xcolor}
\usepackage{blkarray}
\usepackage{braket}
\usepackage{tikz}
\usepackage{tikz-3dplot}
%\usetikzlibrary{tikz-3dplot}
\usetikzlibrary{graphs}
\usetikzlibrary{datavisualization}
\usetikzlibrary{datavisualization.formats.functions}
\usepackage{pgfplotstable}
\usepgfplotslibrary{patchplots}

\setbeamercovered{transparent}

\usetheme{Pittsburgh}
%\usetheme{default}

\setbeamertemplate{sidebar right}{}
\setbeamertemplate{footline}[frame number]
%\usefonttheme{professionalfonts}

%\usepackage{sansmathaccent}
%\usepackage{bm}

%\usepackage{unicode-math}
%%\setmainfont[SlantedFont={Latin Modern Roman Slanted},SlantedFeatures={Color=000000},
%%  SmallCapsFont={TeX Gyre Termes},SmallCapsFeatures={Letters=SmallCaps}]{XITS}
%\setmathfont[math-style=ISO,sans-style=upright]{XITS Math}
%\setmathfont[range={\mathcal,\mathbfcal}]{Latin Modern Math}

\usepackage{sfmath}

%\mathversion{sans}

\newcommand{\Tr}{\mathsf{Tr}}

\definecolor{redorange}{rgb}{1.0, .25, .25}
\definecolor{citation}{rgb}{.1, 0.8, .35}
\newcommand\emm[1]{\textcolor{redorange}{{#1}}}
\newcommand\numc[1]{\textcolor{citation}{{\bf #1}}}

%\newcommand\bm[1]{{\mbox{\boldmath $#1$}}}
\newcommand\bm[1]{{\mathbf{#1}}}
%\newcommand\bm[1]{{\bf #1}}
%\newcommand\bm[1]{\ensuremath{\boldsymbol{#1}}}
%\newcommand\bm[1]{{\textbf{\it #1}}}

\title{Composite system and entanglement}
\author{Ryuhei Mori}
%\institute{$\vcenter{\hbox{\includegraphics[width=30pt]{ELC_logo}}}$ Postdoctoral Fellow of ELC\\ $\vcenter{\hbox{\includegraphics[width=20pt]{titech_logo}}}$ Tokyo Institute of Technology}
\institute{Tokyo Institute of Technology}
\date{}



\begin{document}
\begin{frame}[plain]
\maketitle
\end{frame}


\begin{frame}{Composite system}
\begin{itemize}
\setlength{\itemsep}{2em}
\item System = Set of states \& set of measurements
\item \emm{Composite} system = ``Product'' of systems.
\item \emm{Composite} system of a system of a coin (two-dimentional classical system) and a system of a dice (six-dimentional classical system) is twelve-dimentional classical system.
\item What is a \emm{composite} system of quantum systems ?
\end{itemize}
\end{frame}

\begin{frame}{Tensor product of linear spaces}
For linear product spaces $V$ and $W$ over a field $F$ (usually $\mathbb{R}$ or $\mathbb{C}$),
a tensor space $V\otimes W$ is a linear space spanned by $v\otimes w$ for all $v\in V,\, w\in W$.

\vspace{1em}
\begin{itemize}
\setlength{\itemsep}{2em}
%\item $\forall v\in V,\, \forall w\in W,\, v\otimes w\in V\otimes W$.
\item $\forall c\in F,\,\forall v\in V,\, \forall w\in W,\, c(v\otimes w) = (cv)\otimes w = v\otimes (cw) $.
\item $\forall u,v\in V,\, \forall w\in W,\, (u+v)\otimes w = u\otimes w + v\otimes w $.
\item $\forall v\in V,\, \forall w,y\in W,\, v\otimes (w+y) = v\otimes w + v\otimes y $.
\end{itemize}

\vspace{1em}
Let $(e_i)_i$ be an orthonormal basis of $V$ and $(f_j)_j$ be an orthonormal basis of $W$.
Since $v\otimes w = (\sum_i v_i e_i)\otimes(\sum_j w_j f_j) = \sum_{i,j} v_i w_j (e_i\otimes f_j)$

This implies \emm{$\dim(V\otimes W) =\dim(V)\dim(W)$}.

\vspace{1em}
If $V$ and $W$ are inner product spaces, $V\otimes W$ is also a inner product space defined by
\begin{equation*}
\langle v\otimes w, u\otimes y\rangle =
\langle v, u\rangle\langle w, y\rangle.
\end{equation*}
\end{frame}

\begin{frame}{Vector representation in tensor product}
Let $V:=\mathbb{C}^n,\,W:=\mathbb{C}^m$.
\small
\begin{align*}
e_i&:=\begin{blockarray}{[c]c}0&1\\\vdots\\0&i-1\\1&i\\0&i+1\\\vdots\\0&n\end{blockarray}\in\mathbb{C}^n,&
f_j&:=\begin{blockarray}{[c]c}0&1\\\vdots\\0&j-1\\1&j\\0&j+1\\\vdots\\0&m\end{blockarray}\in\mathbb{C}^m&
\end{align*}
\begin{align*}
e_i\otimes f_j&=\begin{blockarray}{[c]c}0&(1,1)\\\vdots\\0&(i,j-1)\\1&(i,j)\\0&(i,j+1)\\\vdots\\0&(n,m)\end{blockarray}\in\mathbb{C}^n\otimes\mathbb{C}^m
\end{align*}
\end{frame}

\begin{frame}{Vector representation in tensor product}
Let $V:=\mathbb{C}^n,\,W:=\mathbb{C}^m$.
\small
\begin{align*}
e_i\otimes f_j&=\begin{blockarray}{[c]c}0&(1,1)\\\vdots\\0&(i,j-1)\\1&(i,j)\\0&(i,j+1)\\\vdots\\0&(n,m)\end{blockarray}\in\mathbb{C}^n\otimes\mathbb{C}^m
\end{align*}
\begin{align*}
v\otimes w&=\left(\sum_i v_i e_i\right)\otimes\left(\sum_j w_j f_j\right) = \sum_{i,j} v_i w_j (e_i\otimes f_j)\\
&=
\begin{blockarray}{[c]c}\vdots\\v_iw_j&(i,j)\\\vdots\end{blockarray}=\begin{blockarray}{[c]}v_1 w\\v_2 w\\\vdots\\v_nw\end{blockarray}\in\mathbb{C}^n\otimes\mathbb{C}^m
\end{align*}
\end{frame}


\begin{frame}{Linear spaces}
\begin{itemize}
\setlength{\itemsep}{2em}
\item $\mathcal{L}(V,W)$: A linear space spanned by linear maps from a linear space $V$ to a linear space $W$.
\item $\mathcal{L}(V) := \mathcal{L}(V,V)$.
\item $\mathcal{H}(V)$: A real linear space spanned by Hermitian operators acting on a complex linear space $V$.
%\item $\mathcal{J}(V,W) := \mathcal{L}(H(V),H(W))$.
\end{itemize}
\end{frame}

\begin{frame}{Tensor product of linear maps}
\begin{equation*}
\mathcal{L}(V, X)\otimes \mathcal{L}(W, Y)\cong
\mathcal{L}(V\otimes W, X\otimes Y)
\end{equation*}
since the both sides are complex linear spaces with dimension
\begin{equation*}
\dim(V)\dim(W)\dim(X)\dim(Y).
\end{equation*}
%For $A\in \mathcal{L}(V,X),\,B\in \mathcal{L}(W,Y)$,\\
% $A\otimes B\in \mathcal{L}(V, X)\otimes \mathcal{L}(W, Y)$  can be regarded as an element of $\mathcal{L}(V\otimes W, X\otimes Y)$.
% $A\otimes B\in \mathcal{L}(V\otimes W, X\otimes Y)$ is defined by
A natural choice of an isomorphism is 
\begin{align*}
\mathcal{L}(V, X)\otimes \mathcal{L}(W, Y)&\longrightarrow \mathcal{L}(V\otimes W, X\otimes Y)\\
A\otimes B &\longmapsto (\emm{v\otimes w \mapsto A(v)\otimes B(w)}).
\end{align*}

\begin{equation*}
A\otimes B=
\begin{bmatrix}
A_{11}B & A_{12} B &\dotsc & A_{1m}B\\
A_{21}B & A_{22} B &\dotsc & A_{2m}B\\
\vdots & \vdots & \vdots & \vdots\\
A_{n1}B & A_{n2} B &\dotsc & A_{nm}B\\
\end{bmatrix}
\end{equation*}

\end{frame}

\begin{frame}{Tensor product of Hermitian maps}
\begin{equation*}
\mathcal{H}(V)\otimes \mathcal{H}(W)\cong
\mathcal{H}(V\otimes W)
\end{equation*}
since the both sides are real linear spaces with dimension
\begin{equation*}
\dim(V)^2\dim(W)^2.
\end{equation*}
A natural choice of an isomorphism is 
\begin{align*}
\mathcal{H}(V)\otimes \mathcal{H}(W)&\longrightarrow \mathcal{H}(V\otimes W)\\
A\otimes B &\longmapsto (\emm{v\otimes w \mapsto A(v)\otimes B(w)}).
\end{align*}
\end{frame}


\begin{frame}{Composite quantum system}
A quantum system on a complex linear space $\emm{V}$:
\begin{itemize}
\item $\text{Set of states} = \left\{\omega\in \mathcal{H}(\emm{V}) \mid \omega\in C_{\succeq 0}, \Tr(\omega) = 1\right\}$.
\item $\text{Set of binary measurements} = \left\{e\in \mathcal{H}(\emm{V}) \mid e\in C_{\succeq 0}, I-e \in C_{\succeq 0} \right\}$.
\end{itemize}

\vspace{3em}
For a quantum systems on $V$ and $W$, 
a composite system is a quantum system on $V\otimes W$.

\vspace{2em}
A useful formula.
\begin{align*}
\Tr(A\otimes B) &= \sum_{i,j}\bra{i}\otimes\bra{j}A\otimes B \ket{i}\otimes\ket{j}\\
 &= \sum_{i,j}\bra{i}A\ket{i}\bra{j}B\ket{j}\\
&= \Tr(A)\Tr(B)
\end{align*}
\end{frame}

\begin{frame}{Tensor product of states}
\small
\begin{align*}
&\left(\ket{\psi}\bra{\psi}
\otimes
\ket{\phi}\bra{\phi}\right) \left(\ket{v}\otimes \ket{w}\right)\\
&=
\left(\ket{\psi}\bra{\psi}\ket{v}\right)\otimes
\left(\ket{\phi}\bra{\phi}\ket{w}\right)\\
&=\braket{\psi|v}\braket{\phi|w}
\ket{\psi}\otimes \ket{\phi}
\end{align*}
On the other hand,
\begin{align*}
&\left(\ket{\psi}\otimes\ket{\phi}\right)
\left(\bra{\psi}\otimes\bra{\phi}\right) \left(\ket{v}\otimes \ket{w}\right)\\
&=\braket{\psi|v}\braket{\phi|w}
\ket{\psi}\otimes \ket{\phi}
\end{align*}
Hence,
\begin{align*}
&\ket{\psi}\bra{\psi}
\otimes
\ket{\phi}\bra{\phi} =
\left(\ket{\psi}\otimes\ket{\phi}\right)
\left(\bra{\psi}\otimes\bra{\phi}\right).
\end{align*}

\vspace{1em}
We use the notations $\ket{\psi\phi}:=\ket{\psi}\ket{\phi}:=\ket{\psi}\otimes\ket{\phi}$.

For quantum states $\rho$ and $\sigma$
\begin{align*}
\rho\otimes\sigma
&=
\left(\sum_j \mu_j \ket{\psi_j}\bra{\psi_j}\right)\otimes
\left(\sum_k \nu_k \ket{\phi_k}\bra{\phi_k}\right)\\
&=\sum_{j,k}\mu_j\nu_k\ket{\psi_j\phi_k}\bra{\psi_j\phi_k}
\succeq0
\end{align*}
\end{frame}

\begin{frame}{Examples: two-qubit system}
Examples of states
\begin{itemize}
\setlength{\itemsep}{2em}
\item $\ket{0}\bra{0}\otimes \ket{1}\bra{1} = \ket{01}\bra{01}$
\item $\frac12 (\ket{0}\bra{0}\otimes \ket{0}\bra{0} + \ket{0}\bra{0}\otimes \ket{1}\bra{1}) =
 \ket{0}\bra{0}\otimes \frac12(\ket{0}\bra{0} + \ket{1}\bra{1}) = \ket{0}\bra{0}\otimes \frac12 I$.
\item $\frac12 (\ket{1}\bra{1}\otimes \ket{0}\bra{0} + \ket{0}\bra{0}\otimes \ket{1}\bra{1})$.
\item $\frac12(\ket{0}\bra{0}\otimes \ket{0}\bra{0} + \ket{0}\bra{1}\otimes \ket{0}\bra{1}  + \ket{1}\bra{0}\otimes \ket{1}\bra{0} + \ket{1}\bra{1}\otimes \ket{1}\bra{1} )
= \ket{\Phi}\bra{\Phi}$ for $\ket{\Phi}:=\frac1{\sqrt{2}}(\ket{0}\otimes\ket{0}+\ket{1}\otimes\ket{1})$.
\end{itemize}
\end{frame}

\if0
\begin{frame}{Composite system}
Linear space of composite system is regarded as a \emm{tensor product} of linear spaces for the single systems
(justified by two postulates, no-signaling and tomographic locality [Barrett 2007 \numc{275}]).
\begin{align*}
C_1 \otimes_{\min} C_2
&:=\Bigg\{\omega \in V_1\otimes V_2 \mid \omega = \sum_i \lambda_i \omega^{(1)}_i\otimes \omega^{(2)}_i,\\
&\qquad \lambda_i \ge 0,\, \omega^{(1)}_i \in C_1,\, \omega^{(2)}_i \in C_2\Bigg\}\\
C_1 \otimes_{\max} C_2
&:=(C^*_1 \otimes_{\min} C^*_2 )^*
\end{align*}

\vspace{2em}
\begin{equation*}
(C_1 \otimes_{\max} C_2)
\supseteq
C_{1,2}\supseteq
(C_1 \otimes_{\min} C_2)
\end{equation*}
\end{frame}
\fi

%\begin{frame}{Tomography}
%For given density operator $\rho\in D()$.
%\end{frame}


\begin{frame}{Separable states \& entangled states}
A quantum state $\rho$ in a composite system is said to be \emm{separable} if
\begin{align*}
\rho = \sum_{i}p_i \rho_1^i \otimes \rho_2^i
\end{align*}
for some probability distribution $p$ and quantum states $\{\rho_1^i\}$ and $\{\rho_2^i\}$ for subsystems.

\vspace{2em}
If a quantum state is not separable, the state is said to be \emm{entangled} state.

\vspace{2em}
In general, it is difficult to determine whether given state is separable or entangled.
\end{frame}

\begin{frame}{Pure separable states}
\begin{lemma}
A pure state $\ket{\psi}\in V\otimes W$ is separable if and only if there exist pure states $\ket{\varphi}\in V$ and $\ket{\phi}\in W$ such that $\ket{\psi}=\emm{\ket{\varphi}\ket{\phi}}$.
\end{lemma}
\begin{proof}
\small
\begin{align*}
\ket{\psi}\bra{\psi} &= \sum_i p_i \rho_i\otimes\sigma_i\\
&= \sum_i p_i \left(\sum_j \lambda_{i,j} \ket{\varphi_{i,j}}\bra{\varphi_{i,j}}\right)\otimes\left(\sum_k \gamma_{i,k} \ket{\phi_{i,k}}\bra{\phi_{i,k}}\right)\\
&= \sum_\ell q_\ell \ket{\varphi_{\ell}}\bra{\varphi_{\ell}}\otimes\ket{\phi_\ell}\bra{\phi_\ell}
\end{align*}
\begin{align*}
1&=\Tr\left(\ket{\psi}\bra{\psi} \left(\sum_i p_i \rho_i\otimes\sigma_i\right)\right)\\
&= \sum_\ell q_\ell \left|\bra{\psi}\left(\ket{\varphi_{\ell}}\ket{\phi_\ell}\right)\right|^2
\iff
\ket{\psi} = \mathsf{e}^{i\theta_\ell}\ket{\varphi_{\ell}}\ket{\phi_\ell} \qquad\forall\ell\qedhere
\end{align*}
\end{proof}
\end{frame}

\begin{frame}{\large Isomorphism between $V\otimes W$ and $\mathcal{L}(W, V)$}
We consider isomporphism $\mathcal{\emm{M}}$ between $V\otimes W$ and $\mathcal{L}(W, V)$ defined by
\begin{align*}
\mathcal{\emm{M}}:  V\otimes W &\to \mathcal{L}(W,V)\\
\ket{i}_V \ket{j}_W&\mapsto \ket{i}_V\bra{j}_W
\end{align*}
where $(\ket{i}_V)_i$ and $(\ket{j}_W)_j$ are orthonormal basis of $V$ and $W$, respectively.

\begin{align*}
&\mathcal{\emm{M}}\left(\ket{\psi}_V\ket{\varphi}_W\right)\\
&=\mathcal{\emm{M}}\left(\left(\sum_i \psi_i\ket{i}_V\right)\otimes\left(\sum_j \varphi_j\ket{j}_W\right)\right)\\
&=\sum_{i,j}\psi_i\varphi_j \mathcal{\emm{M}}\left(\ket{i}_V\ket{j}_W\right)\\
&=\sum_{i,j}\psi_i\varphi_j \ket{i}_V\bra{j}_W\\
&=\left(\sum_{i}\psi_i\ket{i}_V\right)\left(\sum_j \varphi_j\bra{j}_W\right)
=\ket{\psi}_V\bra{\varphi}_W^*
\end{align*}
\end{frame}

\begin{frame}{Determine the separability of pure state}
\begin{align*}
\ket{\psi}\in V\otimes W \text{ is separable}
&\iff \ket{\psi} = \ket{\varphi}\ket{\phi} \text{ for some } \ket{\varphi}\in V, \ket{\phi}\in W\\
&\iff \mathcal{M}(\ket{\psi}) = \ket{\varphi}\bra{\phi} \text{ for some } \ket{\varphi}\in V, \ket{\phi}\in W\\
&\iff \mathcal{M}(\ket{\psi}) \text{ is \emm{rank 1}}
\end{align*}
\begin{align*}
&\mathcal{M}\left(\frac1{\sqrt{2}} \left(\ket{0}\ket{0} + \ket{1}\ket{1}\right)\right)\\
&=\frac1{\sqrt{2}} \left(\ket{0}\bra{0} + \ket{1}\bra{1}\right) = \frac1{\sqrt{2}} I
\end{align*}

\vspace{2em}
\centering
\large
$\frac1{\sqrt{2}} \left(\ket{0}\ket{0} + \ket{1}\ket{1}\right)$ is entangled!
\end{frame}

\begin{frame}{Scmidt decomposition}
\begin{theorem}[Schmidt decomposition]
For any pure state $\ket{\psi}\in V\otimes W$, there exist \emm{orthonormal systems} $(\ket{v_i})_i$ of $V$ and $(\ket{w_j})_j$ of $W$,
and positive real numbers $(\lambda_i)_i$ such that
\begin{equation*}
\ket{\psi}=\sum_i\lambda_i \ket{v_i}_V\ket{w_i}_W.
\end{equation*}
\end{theorem}
\begin{proof}
Let $A:=\mathcal{M}(\ket{\psi})$.
By the \emm{singular value decomposition},
\begin{align*}
A &= \sum_i \lambda_{i} \ket{s_i}_V\bra{t_i}_W
\end{align*}
Since $\ket{\psi} = \mathcal{M}^{-1}(A)$,
\begin{align*}
\ket{\psi} &= \sum_i \lambda_{i} \ket{s_i}_V\ket{t_i}_W^*.\qedhere
\end{align*}
\end{proof}
\vspace{-.5em}
The number of the terms in the decomposition is called the \emm{Schmidt rank}.
\end{frame}

\begin{frame}{Measurements on composite system}
Set of measurements on the composite sytem on $\emm{V\otimes W}$ is
\begin{align*}
\{(e_1,\dotsc,e_k) \in\mathcal{H}(\emm{V\otimes W})&\mid e_1+\dotsb+e_k=I, e_j\in C_{\succeq 0}\\
&i=1,2,\dotsc,k,\, k=1,2,\dotsc\}.
\end{align*}
For measurements $(P_a\in\mathcal{H}(V))_a$ and $(Q_b\in\mathcal{H}(W))_b$ for the partial systems,
$(P_a\otimes Q_b\in\mathcal{H}(V\otimes W))_{a,b}$ is a measurement since
$P_a\otimes Q_b\succeq 0$ and
\begin{equation*}
\sum_{a,b}P_a\otimes Q_b= \left(\sum_a P_a\right)\otimes \left(\sum_b Q_b\right) = I_V\otimes I_W = I_{V\otimes W}.
\end{equation*}
\end{frame}

\if0
\begin{frame}{Local tomography}
For measurements $\{P_a\}_a$ of quantum system on $V$
and $\{Q_b\}_b$ of quantum system on $W$,
a measurement $\{P_a\otimes Q_b\}_{a,b}$ in the composite system is said to be \emm{local}.

\vspace{3em}
\end{frame}
\fi

\begin{frame}{Partial trace and reduced density matrix}
A probability of outcome of local measurement in a composite system is
\begin{align*}
P_{V\otimes W}(a, b) = \Tr(\rho (P_a \otimes Q_b)).
\end{align*}
\begin{align*}
P_V(a) = \sum_b P_{V\otimes W}(a, b) &= \sum_b \Tr(\rho (P_a \otimes Q_b))\\
 &=  \Tr\left(\rho \left(P_a \otimes \sum_b Q_b\right)\right)\\
 &=  \Tr\left(\rho \left(P_a \otimes I\right)\right)\\
 &=  \Tr(\Tr_W(\rho) P_a).
\end{align*}
The \emm{partial trace} $T_W(\rho)\in\mathcal{H}(V)$ is defined by
\begin{equation*}
\Tr(\Tr_W(\rho) P)
=
\Tr\left(\rho \left(P \otimes I\right)\right)
\end{equation*}
for any $P\in\mathcal{H}(V)$.
Indeed, $\Tr_W(\cdot)$ is a linear operator from $\mathcal{H}(V\otimes W)$ to $\mathcal{H}(V)$ defined by $\Tr_W(\rho_V\otimes \sigma_W)=\Tr(\sigma_W)\rho_V$.
\end{frame}

\begin{frame}{Reduced density matrix from the Schmidt decomposition}
For a pure state $\ket{\psi}\in\mathcal{H}(V)\otimes\mathcal{H}(W)$ with the \emm{Schmidt decomposition}
\begin{equation*}
\ket{\psi}=\sum_i\lambda_i \ket{v_i}_V\ket{w_i}_W
\end{equation*}
it is easy to derive a \emm{reduced density matrices}.

\begin{align*}
\ket{\psi}\bra{\psi}&=\sum_{i,j}\lambda_i\lambda_j \ket{v_i}_V\ket{w_i}_W\bra{v_j}_V\bra{w_j}_W\\
&=\sum_{i,j}\lambda_i\lambda_j \ket{v_i}_V\bra{v_j}_V\otimes \ket{w_i}_W\bra{w_j}_W
\end{align*}
\begin{align*}
\Tr_W(\ket{\psi}\bra{\psi})&=\sum_{i,j}\lambda_i\lambda_j \ket{v_i}_V\bra{v_j}_V\Tr(\ket{w_i}_W\bra{w_j}_W)\\
%&=\sum_{i,j}\lambda_i\lambda_j \ket{v_i}_V\bra{v_j}_V\delta_{i,\,j}\\
&=\sum_i\lambda_i^2 \ket{v_i}_V\bra{v_i}_V\\
\Tr_V(\ket{\psi}\bra{\psi}) &=\sum_i\lambda_i^2 \ket{w_i}_W\bra{w_i}_W
\end{align*}
\end{frame}

\if0
\begin{frame}{Reduced state of a pure state is not necessarily pure}
A two-qubit pure state (called Bell state, Bell pair or EPR pair)
\begin{align*}
\ket{\varphi} := \frac1{\sqrt{2}}(\ket{00}+\ket{11})
\end{align*}
\begin{align*}
\ket{\varphi}\bra{\varphi}&=\frac12(\ket{0}\bra{0}\otimes \ket{0}\bra{0} + \ket{0}\bra{1}\otimes \ket{0}\bra{1} \\
&\qquad + \ket{1}\bra{0}\otimes \ket{1}\bra{0} + \ket{1}\bra{1}\otimes \ket{1}\bra{1} )
\end{align*}
By taking the partial trace for the second qubit, we obtain a reduced density matrix $I/2$.
\end{frame}
\fi

%\begin{frame}{No-cloning theorem}
%\end{frame}

\if0
\begin{frame}{Scmidt decomposition}
\begin{theorem}[Schmidt decomposition]
For any pure state $\ket{\psi}\in V\otimes W$, there exist orthonormal basis $\{\ket{v_i}\}$ of $V$ and $\{\ket{w_i}\}$ of $W$ such that
\begin{equation*}
\ket{\psi}=\sum_i\lambda_i \ket{v_i}\ket{w_i}.
\end{equation*}
\end{theorem}
\begin{proof}[Sketch of a proof]
By a natural isomorphism
\begin{align*}
\Phi:& V\otimes W \to L(W,V)\\
%& v\otimes w \mapsto (u\mapsto \langle w, u\rangle v)
&\ket{v}\ket{w}\mapsto \ket{v}\bra{w}
\end{align*}
the Schmidt decomposition for $V\otimes W$ corresponds to the singular value decomposition for $L(W, V)$.
\end{proof}
\end{frame}
\fi

\if0
\begin{frame}{Spectral decomposition of Hermitian operator}
For any $H\in H(\mathbb{C}^n)$ and $v\in \mathbb{C}^n$,
\begin{equation*}
v,\, Hv,\, H^2v,\,\dotsc,H^nv
\end{equation*}
are linearly dependent.
\begin{align*}
0 &= a_0v+a_1 Tv+\dotsb+a_nT^nv\\
 &= c(T-\lambda_0 I)(T-\lambda_1 I)\dotsm(T-\lambda_n I)v
\end{align*}
which means at least one $T-\lambda_i I$ is not injective.
\begin{equation*}
\lambda \langle v, v\rangle
=\langle \lambda v, v\rangle
=\langle H v, v\rangle
=\langle v, H v\rangle
=\langle v, \lambda v\rangle
=\lambda^* \langle v, v\rangle
\end{equation*}
\begin{equation*}
\lambda \langle v, w\rangle
=\langle \lambda v, w\rangle
\end{equation*}

\end{frame}

\begin{frame}{Purification}
\begin{theorem}
For any density matrix $\rho$ on $V$, there exists a pure state $\ket{\psi}$ of a composite system on $V\otimes W$ for some $W$ such that $\Tr_W(\ket{\psi}\bra{\psi})=\rho$.
\end{theorem}
\end{frame}
\fi

\if0
\begin{frame}{Superdense coding}
Alice can send \emm{two} bits to Bob by sending a single qubit and using a shared Bell state.
\begin{align*}
\ket{\varphi_{00}} &= \frac1{\sqrt{2}}(\ket{00}+\ket{11})\\
\ket{\varphi_{01}} &= \frac1{\sqrt{2}}(\ket{10}+\ket{01}),& \text{by $X$}\\
\ket{\varphi_{10}} &= \frac1{\sqrt{2}}(\ket{00}-\ket{11}),& \text{by $Z$}\\
\ket{\varphi_{11}} &= \frac1{\sqrt{2}}(\ket{10}-\ket{01}),& \text{by $XZ$}\\
\end{align*}
These are \emm{orthogonal}.
\end{frame}
\fi



\begin{frame}{Assignments}
\begin{enumerate}
\setlength{\itemsep}{2em}
%\item Show $A\otimes B \succeq 0$ for any $A\succeq 0$ and $B\succeq 0$.
\item Show the Schmidt decomposition of the following pure states
\begin{enumerate}[A]
\setlength{\itemsep}{1em}
\item $\frac12(\ket{00}+\ket{01}+\ket{10}+\ket{11})$
\item $\frac12(\ket{00}-\ket{01}-\ket{10}+\ket{11})$
\item $\frac12(\ket{00}-i\ket{01}+i\ket{10}+\ket{11})$
\item $\cos\theta\ket{00}-\sin\theta\ket{01}+\sin\theta\ket{10}+\cos\theta\ket{11}$
\end{enumerate}
\item Show the reduced density matrices for each qubit of the above pure states.
\end{enumerate}
\end{frame}

\end{document}
