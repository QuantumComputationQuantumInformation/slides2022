\documentclass{beamer}
%\usepackage{xspace}
\usepackage{amsmath,amssymb}
\usepackage{graphicx}
%\usepackage{svg}
%\usepackage{pgfpages}
%\pgfpagesuselayout{4 on 1}[a4paper,border shrink=5mm,landscape]
%\usepackage{psfrag}
%\usepackage[usenames,dvipsnames]{xcolor}
\usepackage{braket}
\usepackage{qcircuit}
\usepackage{tikz}
\usetikzlibrary{circuits.logic.US}
\usetikzlibrary{graphs}
\usetikzlibrary{datavisualization}
\usetikzlibrary{datavisualization.formats.functions}
\usepackage{pgfplotstable}
\usepgfplotslibrary{patchplots}

\setbeamercovered{transparent}

\usetheme{Pittsburgh}
%\usetheme{default}

\setbeamertemplate{sidebar right}{}
\setbeamertemplate{footline}[frame number]
%\usefonttheme{professionalfonts}

%\usepackage{sansmathaccent}
%\usepackage{bm}

%\usepackage{unicode-math}
%%\setmainfont[SlantedFont={Latin Modern Roman Slanted},SlantedFeatures={Color=000000},
%%  SmallCapsFont={TeX Gyre Termes},SmallCapsFeatures={Letters=SmallCaps}]{XITS}
%\setmathfont[math-style=ISO,sans-style=upright]{XITS Math}
%\setmathfont[range={\mathcal,\mathbfcal}]{Latin Modern Math}

\usepackage{sfmath}

%\mathversion{sans}

\newcommand{\Tr}{\mathsf{Tr}}

\definecolor{redorange}{rgb}{1.0, .25, .25}
\definecolor{citation}{rgb}{.1, 0.8, .35}
\newcommand\emm[1]{\textcolor{redorange}{{#1}}}
\newcommand\numc[1]{\textcolor{citation}{{\bf #1}}}

%\newcommand\bm[1]{{\mbox{\boldmath $#1$}}}
\newcommand\bm[1]{{\mathbf{#1}}}
%\newcommand\bm[1]{{\bf #1}}
%\newcommand\bm[1]{\ensuremath{\boldsymbol{#1}}}
%\newcommand\bm[1]{{\textbf{\it #1}}}

\title{Quantum state discrimination and Holevo--Helstrom theorem}
\author{Ryuhei Mori}
%\institute{$\vcenter{\hbox{\includegraphics[width=30pt]{ELC_logo}}}$ Postdoctoral Fellow of ELC\\ $\vcenter{\hbox{\includegraphics[width=20pt]{titech_logo}}}$ Tokyo Institute of Technology}
\institute{Tokyo Institute of Technology}
\date{}



\begin{document}
\begin{frame}[plain]
\maketitle
\end{frame}



\begin{frame}{Discrimination of probability distributions}
\emm{Discrimination} of probability distributions:

\vspace{1em}
\begin{itemize}
\item $p_0,\,p_1$: Known probability distributions on a finite set $\mathcal{X}$.
\item $\lambda\in[0,1]$: A probability of choosing the probability distributions.
\end{itemize}

\vspace{2em}
\begin{itemize}
\setlength{\itemsep}{2em}
\item Input: $x\in\mathcal{X}$ drawn from $p_0$ with probability $\lambda$, and from $p_1$ with probability $1-\lambda$.
\item Output: $i\in\{0,1\}$ that indicates the probability distribution $p_i$.
\end{itemize}
\end{frame}

\begin{frame}{Maximum probability of success: Discrimination of probability distributions}
\small
\begin{theorem}
The maximum probability of success is equal to
\begin{equation*}
\frac{1+\emm{\|\lambda p_0 - (1-\lambda)p_1\|_1}}2
\end{equation*}
where
$\|a\|_1 := \sum_{x} |a(x)|$.
\end{theorem}
\begin{proof}
Let $d\colon\mathcal{X}\to\{0, 1\}$ be a discriminator.
Assume we get 1 point if succeeds, and loose 1 point if fails.
Then, expected point is bounded by
\begin{align*}
&\lambda \sum_{x\in\mathcal{X}} p_0(x) (-1)^{d(x)} +
(1-\lambda) \sum_{x\in\mathcal{X}} p_1(x) (-1)^{d(x)+1}\\
&=\sum_{x\in\mathcal{X}} \left(\lambda p_0(x) - (1-\lambda)p_1(x)\right) (-1)^{d(x)} \le
\|\lambda p_0 - (1-\lambda)p_1\|_1.
\end{align*}
The equality is achieved by $d(x)$ that is 0 iff $\lambda p_0(x) \ge (1-\lambda)p_1(x)$.
On the other hand, the expected point is $p_\mathsf{succ} - p_\mathsf{fail} = 2 p_\mathsf{succ} - 1$.
\end{proof}
\end{frame}

\begin{frame}{Discrimination of quantum states}
Discrimination of \emm{quantum} states:

\vspace{1em}
\begin{itemize}
\item $\rho_0,\,\rho_1$: Known quantum states.
\item $\lambda\in[0,1]$: A probability of choosing the quantum states.
\end{itemize}

\vspace{2em}
\begin{itemize}
\setlength{\itemsep}{2em}
\item Input: A \emm{quantum} state $\rho_0$ is given with probability $\lambda$, and $\rho_1$ is given with probability $1-\lambda$.
\item Output: $i\in\{0,1\}$ that indicates the given state $\rho_i$.
\end{itemize}
\end{frame}

\begin{frame}{Maximum probability of success: Discrimination of quantum states}
\small
\begin{theorem}[Holevo--Helstrom theorem]
The maximum probability of success is equal to
\begin{equation*}
\frac{1+\emm{\|\lambda \rho_0 - (1-\lambda)\rho_1\|_1}}2
\end{equation*}
where
$\|A\|_1 := \Tr(\sqrt{A^\dagger A})$, which is a sum of the singular values of $A$.
\end{theorem}
\begin{proof}
Once a measurement $(P_y)_{y\in\mathcal{Y}}$ is fixed, we get a classical probability distribution $p_0(y) := \Tr(\rho_0 P_y)$ and $p_1(y) := \Tr(\rho_1 P_y)$.
In this case, the maximum probability of success is given by
\begin{equation*}
\frac{1+\|\lambda p_0 - (1-\lambda)p_1\|_1}2
\end{equation*}
Hence, it's sufficient to show
\begin{equation*}
%\max_{(P_x)_{x\in\mathcal{X}}} \left\|\lambda \Tr(\rho_0 P_x) - (1-\lambda)\Tr(\rho_1 P_x)\right\|_1
\max_{(P_y)_{y\in\mathcal{Y}}} \left\|\lambda p_0 - (1-\lambda)p_1\right\|_1
=
\|\lambda \rho_0 - (1-\lambda)\rho_1\|_1.
\end{equation*}
\end{proof}
\end{frame}

\begin{frame}{Holevo--Helstrom theorem}
\scriptsize\vspace{-1em}
\begin{align*}
%\max_{(P_x)_{x\in\mathcal{X}}} \left\|\lambda \Tr(\rho_0 P_x) - (1-\lambda)\Tr(\rho_1 P_x)\right\|_1
&\max_{(P_y)_{y\in\mathcal{Y}}} \left\|\lambda p_0 - (1-\lambda)p_1\right\|_1
= \max_{(P_y)_{y\in\mathcal{Y}}} \sum_{y\in\mathcal{Y}}\left|\lambda p_0(y) - (1-\lambda)p_1(y)\right|\\
&= \max_{(P_y)_{y\in\mathcal{Y}}} \sum_{y\in\mathcal{Y}}\left|\lambda \Tr(\rho_0 P_y) - (1-\lambda)\Tr(\rho_1 P_y)\right|\\
&= \max_{(P_y)_{y\in\mathcal{Y}}} \sum_{y\in\mathcal{Y}}\left|\Tr\left((\lambda\rho_0 - (1-\lambda)\rho_1) P_y\right)\right|
\end{align*}
Let
\begin{align*}
\lambda \rho_0 - (1-\lambda)\rho_1 = \sum_{x\in\mathcal{X}} \mu_x \ket{\psi_x}\bra{\psi_x}
\end{align*}
be a spectral decomposition. Then,
\begin{align*}
&\max_{(P_y)_{y\in\mathcal{Y}}} \sum_{y\in\mathcal{Y}}\left|\Tr\left((\lambda\rho_0 - (1-\lambda)\rho_1) P_y\right)\right|
=\max_{(P_y)_{y\in\mathcal{Y}}} \sum_{y\in\mathcal{Y}}\left|\sum_{x\in\mathcal{X}}\mu_x \bra{\psi_x} P_y\ket{\psi_x}\right|\\
&\le\max_{(P_y)_{y\in\mathcal{Y}}} \sum_{y\in\mathcal{Y}}\sum_{x\in\mathcal{X}}|\mu_x| \bra{\psi_x} P_y\ket{\psi_x}
=\sum_{x\in\mathcal{X}} |\mu_x|
=
\|\lambda \rho_0 - (1-\lambda)\rho_1\|_1.
\end{align*}
The maximum is achieved by $\mathcal{Y}=\mathcal{X}$ and $P_x = \ket{\psi_x}\bra{\psi_x}$,\\
and $\mathcal{Y}=\{0,1\}$ and $P_0 = \sum_{x: \mu_x\ge 0} \ket{\psi_x}\bra{\psi_x}$, $P_1 = \sum_{x: \mu_x< 0} \ket{\psi_x}\bra{\psi_x}$.
\end{frame}

\begin{frame}{Discrimination of pure states}
The maximum probability of success for discrimination of $\ket{\psi}\bra{\psi}$ and $\ket{\varphi}\bra{\varphi}$ is given by
\begin{equation*}
\bigl\|\lambda \ket{\psi}\bra{\psi} - (1-\lambda)\ket{\varphi}\bra{\varphi}\bigr\|_1
\end{equation*}
Let $A :=\lambda \ket{\psi}\bra{\psi} - (1-\lambda)\ket{\varphi}\bra{\varphi}$. Then, $A$ has rank at most \emm{two}.
From
\begin{align*}
\Tr(A) &= \mu_0 + \mu_1 = \lambda - (1 - \lambda)\\
\Tr(A^2) &= \mu_0^2 + \mu_1^2 = \lambda^2 + (1-\lambda)^2 - 2\lambda(1-\lambda)|\braket{\psi|\varphi}|^2,
\end{align*}
\begin{align*}
2\mu_0\mu_1 &= (\mu_0 + \mu_1)^2 - (\mu_0^2 + \mu_1^2) = -2\lambda(1-\lambda)(1-|\braket{\psi|\varphi}|^2)\le 0.\\
(\mu_0-\mu_1)^2 &= (\mu_0^2 + \mu_1^2) - 2\mu_0\mu_1 = (\lambda + (1-\lambda))^2 - 4\lambda(1-\lambda)|\braket{\psi|\varphi}|^2\\
\|A\|_1 &= |\mu_0| + |\mu_1| = |\mu_0-\mu_1| = \emm{\sqrt{1 - 4\lambda(1-\lambda)|\braket{\psi|\varphi}|^2}}.
\end{align*}
\end{frame}

%\begin{frame}{Distance measures for classical states to quantum states}
%Let $\delta$ be a distance measure of classical states $p_0$ and $p_1$ that satisfies
%\begin{itemize}
%\item Positivity: $\delta(p_0, p_1)\ge 0$. The equality holds if and only if $p_0=p_1$.
%\item Triangle inequality: $\delta(p_0, p_1)\le \delta(p_0, p_2)+\delta(p_2,p_1)$.
%\end{itemize}
%\end{frame}

%\begin{frame}{Hellinger distance}
%\begin{align*}
%\|\lambda p_0(x) - (1-\lambda)p_1(x)\|_2
%\end{align*}
%\end{frame}

\begin{frame}{Trace norm}
\footnotesize
The trace norm is the sum of the singular values.
Hence, it satisfies the \emm{unitary invariance}
\begin{equation*}
\|UAV\|_1=\|A\|_1
\end{equation*}
for any unitary matrices $U$ and $V$.

For
\begin{align*}
A &= \sum_j \lambda_j \ket{\psi_j}\bra{\varphi_j}\qquad\text{(singular decomposition)}\\
%U &= \sum_j \mathsf{e}^{i\theta_j} \ket{\varphi_j}\bra{\psi_j}\qquad\text{(singular decomposition)}
U &= \sum_j \ket{\tau_j}\bra{\psi_j}\qquad\text{(singular decomposition)}
\end{align*}
\begin{align*}
\left|\Tr(UA)\right|
&= \left|\Tr\left(\sum_j \lambda_j \ket{\tau_j}\bra{\varphi_j}\right)\right|\\
%&\le \sum_j \left|\Tr\left(\mathsf{e}^{i\theta_j}\lambda_j \ket{\varphi_j}\bra{\varphi_j}\right)\right|\\
&\le \sum_j \lambda_j \left|\Tr\left(\ket{\tau_j}\bra{\varphi_j}\right)\right|
= \sum_j \lambda_j |\braket{\varphi_j|\tau_j}|
\le \sum_j \lambda_j = \|A\|_1
\end{align*}
By setting $\ket{\tau_j}=\ket{\varphi_j}$, the equalities are satisfied. We obtain \emm{$\max_U|\Tr(UA)| = \|A\|_1$}.
Hence, the trace norm satisfies \emm{the triangle inequality}.
\begin{align*}
\|A+B\|_1 &= \max_U|\Tr(U(A+B))|
\le \max_U(|\Tr(UA)| + |\Tr(UB)|)\\
&\le \max_U|\Tr(UA)| + \max_U|\Tr(UB)|
=\|A\|_1+\|B\|_1
\end{align*}
\end{frame}

\begin{frame}{Unitary discrimination}
Discrimination of \emm{unitary operators}:

\vspace{2em}
\begin{itemize}
\setlength{\itemsep}{2em}
\item Input: A \emm{unitary operator} $U_0$ is given with probability $\lambda$, and $U_1$ is given with probability $1-\lambda$ as an oracle $\mathcal{O}$ that can be used once.
\item Output: $i\in\{0,1\}$ that indicates the given unitary $U_i$.
\end{itemize}
\end{frame}


\begin{frame}{Unitary discrimination}
If algorithm call the oracle $\mathcal{O}$ on a state $\ket{\psi}$, we get either of $U_0\ket{\psi}$ or $U_1\ket{\psi}$.

Then, the maximum probability of success is given by
\begin{align*}
&\max_{\ket{\psi}}\|\lambda U_0\ket{\psi}\bra{\psi} U_0^\dagger - (1-\lambda) U_1\ket{\psi}\bra{\psi}U_1^\dagger\|_1\\
&=
\max_{\ket{\psi}}\sqrt{1-4\lambda(1-\lambda)|\bra{\psi}U_0^\dagger U_1\ket{\psi}|^2}
\end{align*}
For $V:=U_0^\dagger U_1$, let $V = \sum_x \mu_x \ket{\varphi_x}\bra{\varphi_x}$ be a spectral decomposition. Note that $|\mu_x|=1$ for all x.
Let $\ket{\psi} = \sum_x \alpha_x \ket{\varphi_x}$.

Then, $\bra{\psi}U_0^\dagger U_1\ket{\psi}= \sum_x |\alpha_x|^2 \mu_x$, which is a convex combination of $(\mu_x)_x$.
Let $\emm{\theta_{\mathsf{cover}}}$ be the smallest angle that covers all eigenvalues $(\mu_x)_x$.
Then, $\min_{\ket{\psi}} |\bra{\psi}U_0^\dagger U_1\ket{\psi}|= \cos(\frac{\emm{\theta_\mathsf{cover}}}2)$ if $\theta_\mathsf{cover}\le \pi$, and 0 otherwise.
\end{frame}


\begin{frame}{The diamond distance}
\small
The diamond distance $D$ is defined by
\begin{align*}
D(A,B)&:=
\max_{\ket{\psi}}\|A\ket{\psi}\bra{\psi} A^\dagger - B\ket{\psi}\bra{\psi}B^\dagger\|_1.
\end{align*}

Then, $D$ is \emm{unitary invariant}, i.e., $D(UAV, UBV) = D(A, B)$ for any unitary matrices $U$ and $V$.

\footnotesize
\begin{align*}
D(A,B)&=
\max_{\ket{\psi}}\|A\ket{\psi}\bra{\psi} A^\dagger - B\ket{\psi}\bra{\psi}B^\dagger\|_1\\
&=\max_{\ket{\psi}}\|A\ket{\psi}\bra{\psi} A^\dagger \emm{- C\ket{\psi}\bra{\psi}C^\dagger+C\ket{\psi}\bra{\psi}C^\dagger}-B\ket{\psi}\bra{\psi}B^\dagger\|_1\\
&\le\max_{\ket{\psi}}(\|A\ket{\psi}\bra{\psi} A^\dagger - C\ket{\psi}\bra{\psi}C^\dagger\|_1+\|C\ket{\psi}\bra{\psi}C^\dagger-B\ket{\psi}\bra{\psi}B^\dagger\|_1)\\
&\le\max_{\ket{\psi}}\|A\ket{\psi}\bra{\psi} A^\dagger - C\ket{\psi}\bra{\psi}C^\dagger\|_1+\max_{\ket{\psi}}\|C\ket{\psi}\bra{\psi}C^\dagger-B\ket{\psi}\bra{\psi}B^\dagger\|_1\\
&=D(A,C)+D(C,B)
\end{align*}
The diamond distance $D$ satisfies the \emm{triangle inequality}.
\end{frame}

\begin{frame}{Assignment}
\begin{enumerate}
\setlength{\itemsep}{2em}
\item Show the maximum probability of success for discriminating $\ket{0}$ and $\ket{+}$ given with the uniform probability.
\item Show a binary optimal measurement for the discrimination of $\ket{0}$ and $\ket{+}$ given with the uniform probability.
\item Show the maximum probability of success for discriminating $I$ and $R_Z(\theta)$ given with the uniform probability.
Show the input state $\ket{\psi}$ for the oracle as well.
\item {[Advanced]} Show the maximum probability of success for discriminating 
$R_{\mathbf{v}}(\theta)$ and $R_{\mathbf{w}}(\eta)$ given with the uniform probability.
\end{enumerate}
\end{frame}

\end{document}
