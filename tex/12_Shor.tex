\documentclass{beamer}
%\usepackage{xspace}
\usepackage{amsmath,amssymb}
\usepackage{graphicx}
%\usepackage{svg}
%\usepackage{pgfpages}
%\pgfpagesuselayout{4 on 1}[a4paper,border shrink=5mm,landscape]
%\usepackage{psfrag}
%\usepackage[usenames,dvipsnames]{xcolor}
\usepackage{braket}
\usepackage{qcircuit}
%\usepackage{algorithmicx}
\usepackage{algpseudocode}
\usepackage{tikz}
\usepackage{tikz-3dplot}
\usetikzlibrary{circuits.logic.US}
\usetikzlibrary{graphs}
\usetikzlibrary{datavisualization}
\usetikzlibrary{datavisualization.formats.functions}
\usepackage{pgfplotstable}
\usepgfplotslibrary{patchplots}

\setbeamercovered{transparent}

\usetheme{Pittsburgh}
%\usetheme{default}

\setbeamertemplate{sidebar right}{}
\setbeamertemplate{footline}[frame number]
%\usefonttheme{professionalfonts}

%\usepackage{sansmathaccent}
%\usepackage{bm}

%\usepackage{unicode-math}
%%\setmainfont[SlantedFont={Latin Modern Roman Slanted},SlantedFeatures={Color=000000},
%%  SmallCapsFont={TeX Gyre Termes},SmallCapsFeatures={Letters=SmallCaps}]{XITS}
%\setmathfont[math-style=ISO,sans-style=upright]{XITS Math}
%\setmathfont[range={\mathcal,\mathbfcal}]{Latin Modern Math}

\usepackage{sfmath}

%\mathversion{sans}

\newcommand{\Tr}{\mathsf{Tr}}

\definecolor{redorange}{rgb}{1.0, .25, .25}
\definecolor{citation}{rgb}{.1, 0.8, .35}
\newcommand\emm[1]{\textcolor{redorange}{{#1}}}
\newcommand\numc[1]{\textcolor{citation}{{\bf #1}}}

%\newcommand\bm[1]{{\mbox{\boldmath $#1$}}}
\newcommand\bm[1]{{\mathbf{#1}}}
%\newcommand\bm[1]{{\bf #1}}
%\newcommand\bm[1]{\ensuremath{\boldsymbol{#1}}}
%\newcommand\bm[1]{{\textbf{\it #1}}}

\title{Shor's algorithm}
\author{Ryuhei Mori}
%\institute{$\vcenter{\hbox{\includegraphics[width=30pt]{ELC_logo}}}$ Postdoctoral Fellow of ELC\\ $\vcenter{\hbox{\includegraphics[width=20pt]{titech_logo}}}$ Tokyo Institute of Technology}
\institute{Tokyo Institute of Technology}
\date{}



\begin{document}
\begin{frame}[plain]
\maketitle
\end{frame}


\begin{frame}{Integer factoring and primality test}
\begin{itemize}
\setlength{\itemsep}{2em}
\item \textsc{PrimalityTest}:\\
Input: $N\in\mathbb{N}$\\
Output: YES if $N$ is a prime number, NO if $N$ is a composite number.
\item \textsc{IntegerFactoring}:\\
Input: $N\in\mathbb{N}$\\
Output: $a\in\mathbb{N}$ satisfying $a\ne 1, N$ and $a$ divides $N$.
\end{itemize}

\vspace{1em}
\begin{center}
\centering
Does there exist an algorithm with time complexity $O((\log N)^c)$ ?
\end{center}

\vspace{1em}
$\textsc{PrimalityTest}\in\mathsf{coRP}$ {\scriptsize [Solovay, Strassen, 1977], \emm{[Miller, Rabin 1980]}}.

$\textsc{PrimalityTest}\in\mathsf{P}$ {\scriptsize [Agrawal, Kayal, Saxena, 2004]}.

It is \emm{believed} that $\textsc{IntegerFactoring}\notin\mathsf{BPP}$.

$\textsc{IntegerFactoring}\in\emm{\mathsf{BQP}}$. { \emm{[Shor 1994]}}
\end{frame}

\begin{frame}{Nontrivial square root of $1$}
\begin{equation*}
x^2 = 1 \mod N
\end{equation*}
\begin{align*}
&x^2 = 1 \mod N\\
\iff& (x-1)(x+1) = 0 \mod N
\end{align*}
If $N$ is a prime number, $x=\pm 1$ are only solutions.

\vspace{2em}
If there exists a \emm{nontrivial square root of 1}, then $N$ must be composite number.

\vspace{2em}
If we find a \emm{nontrivial square root $x$ of 1}, we can also find a nontrivial factor of $N$ by $\mathsf{gcd}(x\pm 1, N)$.
\end{frame}

\begin{frame}{Fermat little theorem}
\begin{theorem}[Fermat little theorem]
If $p$ is a prime number, any integer $a$ that is not a multiple of $p$,
\begin{equation*}
a^{\emm{p-1}}=1\mod p.
\end{equation*}
\end{theorem}
\begin{proof}
The map
\begin{equation*}
x \longmapsto a x \mod p
\end{equation*}
is a bijection on $\{1,\dotsc,p-1\}$.
Hence,
\begin{align*}
\prod_{x=1}^{p-1}x &= \prod_{x=1}^{p-1}(ax)\mod p\\
\iff 1 &= a^{p-1}\mod p.
\end{align*}
\end{proof}
\end{frame}


\begin{frame}{Fermat test}
\begin{algorithmic}
\Function{Fermat}{$N$}
\Loop{\qquad $k$ times}
\State $a\gets \text{a random integer in } [2,N-2]$
\If{$a^{N-1}\ne 1 \mod N$}
\State \Return NO
\EndIf
\EndLoop
\State \Return YES
\EndFunction
\end{algorithmic}

\vspace{3em}
Carmichel numbers ($561 = 3\cdot 11\cdot 17,\, 1105 = 5\cdot 13\cdot 17, \cdots$) passes the Fermat test for \emm{all $a$ coprime with $N$}.
\end{frame}

\begin{frame}{Finding the nontirivial square root of 1}
We assume that $a^{N-1} = 1 \mod N$ for some integer $a\in[2,N-2]$, 

\vspace{2em}
Let $u$ and $d$ be an integer and an odd integer, respectively, satisfying $N-1=2^ud$.

\vspace{2em}
\centering
\renewcommand{\arraystretch}{1.5}
\begin{tabular}{|c|c|c|c|c|c|c|c|c|}
\hline
$a^d$ & $a^{2d}$ & $a^{2^2d}$ & $\cdots$ & $a^{2^{k-1} d}$ & $a^{2^k d}$ & $\cdots$ & $a^{2^{u-1} d}$ & $a^{2^u d}$\\
\hline
* & * & * & $\dotsm$ & \emm{$z\ne 1$} & $1$ & $\dotsm$ & $1$ & $1$\\
\hline
\end{tabular}

\vspace{2em}
$z$ is a square root of 1 modulo $N$.
\end{frame}

\begin{frame}{Miller--Rabin primality test}
\small
\begin{algorithmic}
\Function{Miller--Rabin}{$N$}
\State Let $u$ and $d$ be an integer and an odd integer, respectively, satisfying $N=2^ud+1$
\Loop\qquad $k$ times
\State $a\gets \text{a random integer in } [2,N-2]$
\State $x\gets a^{d}\bmod N$
\If{$x = 1$ \text{ or } $x = N-1$}
\textbf{continue}
\EndIf
\Loop{\qquad $u-1$ times}
\State $x\gets x^2\bmod N$
\If{$x = N-1$}
\textbf{break}
\EndIf
\EndLoop
\If{$x \ne N-1$}
\Return NO
\EndIf
\EndLoop
\State \Return YES
\EndFunction
\end{algorithmic}
\end{frame}

\begin{frame}{Why Miller--Rabin algorithm doesn't solve \textsc{IntegerFactoring}}
In fact, the Miller--Rabin test outputs NO with probability $1-1/4^{k}$ for composite $N$.

\vspace{2em}
The Miller--Rabin algorithm seems to find a nontrivial square root of 1, which means that we can also find a nontrivial factor of $N$, right ?

\vspace{2em}
\centering
NO!

\vspace{2em}
\renewcommand{\arraystretch}{1.5}
\begin{tabular}{|c|c|c|c|c|}
\hline
$a^d$ & $a^{2d}$ & $a^{2^2d}$ & $\dotsm$ & $a^{2^u d}$\\
\hline
* & * & * & $\dotsm$ & $\emm{\ne1}$\\
\hline
\end{tabular}
\end{frame}

\begin{frame}{Shor's algorithm}
\begin{algorithmic}
\Function{Shor}{$N$: An odd integer}
\If{$N=a^b$ for some $a\ge 1$ and $b\ge 2$}
\State \Return a
\EndIf
\Loop
\State $a\gets \text{a random integer in } [2,N-2]$.
\State $b\gets \mathsf{gcd}(a, N)$.
\If{$b\ne 1$}
\Return $b$
\EndIf
\State $r\gets \emm{\textsc{OrderFinding}(a, N)}$.
\If{$r$ is odd}
\textbf{continue}
\EndIf
\If{$a^{r/2}\ne N-1\bmod N$}
\State \Return $\mathsf{gcd}(a^{r/2}+1, N)$
\EndIf
\EndLoop
\EndFunction
\end{algorithmic}
\end{frame}

\begin{frame}{Eigenvalues of the unitary operator}
Let $r$ be the \emm{order} of $a$ modulo $N$, which is a smallest positive integer satisfying
\begin{equation*}
a^r = 1 \mod N.
\end{equation*}
For $a$ that is \emm{coprime with $N$}, define the unitary operator $U_a$ by
\begin{align*}
U_a\ket{x} =
\begin{cases}
\ket{ax \bmod N}& \text{if $x<N$}\\
\ket{x}& \text{Otherwise.}
\end{cases}
\end{align*}

%\begin{itemize}
%\setlength{\itemsep}{1em}
%\item $U_a^r = I$.  This means that all eigenvalues of $U$ are in the form $\mathsf{e}^{2\pi i \frac{s}{r}}$ for $s\in\{0,1,2,\dotsc,r-1\}$.
%\item $U^t \ne I$ for all $t\in\{0,1,2,\dotsc,r-1\}$. This means that there exists an eigenvalue of the form $\mathsf{e}^{2\pi i \frac{s}{r}}$ with $s$ coprime with $r$.
%\end{itemize}
Here, \emm{$U_a^r = I$}.  This means that all eigenvalues of $U_a$ are in the form $\mathsf{e}^{2\pi i \frac{s}{r}}$ for $s\in\{0,1,2,\dotsc,r-1\}$.

\vspace{1em}
By quantum phase estimation for $U_a$, an approximation of $\frac{s}{r}$ can be computed efficiently.

\vspace{1em}
From $0.b_nb_{n-1}\dotsm b_1\approx \frac{s}{r}$, we can extract the denominator $r$ if $s$ is comprime with $r$.
\end{frame}

\begin{frame}{Eigenvectors of the unitary operator}
For $s\in\{0,1,\dotsc,r-1\}$,
\begin{align*}
\ket{\psi_s}&:=\frac1{\sqrt{r}} \sum_{j=0}^{r-1} \mathsf{e}^{-2\pi i \frac{sj}{r}} \ket{a^j \bmod N}.
\end{align*}
\begin{align*}
U_a\ket{\psi_s}&=\frac1{\sqrt{r}} \sum_{j=0}^{r-1} \mathsf{e}^{-2\pi i \frac{sj}{r}} \ket{a^{\emm{j+1}} \bmod N}\\
&=\mathsf{e}^{2\pi i \frac{s}{r}}\frac1{\sqrt{r}} \sum_{j=0}^{r-1} \mathsf{e}^{-2\pi i \frac{s(j+1)}{r}} \ket{a^{j+1} \bmod N}\\
&=\mathsf{e}^{2\pi i \frac{s}{r}}\frac1{\sqrt{r}} \sum_{j=0}^{r-1} \mathsf{e}^{-2\pi i \frac{sj}{r}} \ket{a^j \bmod N}\\
&=\mathsf{e}^{2\pi i \frac{s}{r}}\ket{\psi_s}.
\end{align*}
\end{frame}

\begin{frame}{The uniform sperposition of the eigenvectors}
For $s\in\{0,1,\dotsc,r-1\}$,
\begin{align*}
\ket{\psi_s}&:=\frac1{\sqrt{r}} \sum_{j=0}^{r-1} \mathsf{e}^{-2\pi i \frac{sj}{r}} \ket{a^j \bmod N}.
\end{align*}
\begin{align*}
\frac1{\sqrt{r}}\sum_{s=0}^{r-1}\ket{\psi_s}
&=\frac1{r} \sum_{s=0}^{r-1}\sum_{j=0}^{r-1} \mathsf{e}^{-2\pi i \frac{sj}{r}} \ket{a^j \bmod N}\\
&=\sum_{j=0}^{r-1} \left(\frac1{r} \sum_{s=0}^{r-1}\mathsf{e}^{-2\pi i \frac{sj}{r}}\right) \ket{a^j \bmod N}\\
&=\emm{\ket{1}}
\end{align*}
\end{frame}

\begin{frame}{Quantum phase estimation}
\[
\Qcircuit @C=2em @R=1em {
\lstick{\ket{0}}   &\gate{H} & \ctrl{3} & \qw&\qw & \multigate{2}{\mathsf{IQFT}}&  \qw\\
\lstick{\ket{0}}   &\gate{H} & \qw &  \ctrl{2} &\qw & \ghost{IQFT} & \qw\\
\lstick{\ket{0}}   &\gate{H} & \qw &  \qw & \ctrl{1} & \ghost{IQFT}& \qw\\
\lstick{\ket{1}} &{/}\qw      & \gate{U_a} & \gate{U_a^2} &\gate{U_a^4} &\qw & \qw
}
\]

\vspace{2em}
For uniformly chosen $s\in\{0,1,\dotsc,r-1\}$, we obtain an approximation of $s/r$.
\end{frame}

\begin{frame}{Continued fraction}
\begin{equation*}
\theta=\cfrac{1}{a_1+\cfrac{1}{a_2+\cfrac{1}{a_3+\cfrac{1}{a_4+\dotsb}}}}
\end{equation*}
\begin{theorem}
Suppose $s/r$ is a rational number satisfying
\begin{equation*}
\left|\frac{s}{r}-\theta\right|\le\frac1{2r^2}.
\end{equation*}
Then, $s/r$ is a convergent of the continued fraction for $\theta$.
\end{theorem}
\end{frame}

\begin{frame}{The probability that we can calculate $r$ from $s/r$}
For uniformly chosen $s\in\{0,1,\dotsc,r-1\}$, we obtain an approximation of $s/r$.

%\vspace{1em}
%If $s/r = 0$ is measured, we replace it by $s/r = 1$ that means $s=0$ is replaced by $s=r$.
%Hence, $s$ is chosen uniformly from $\{1,2,\dotsc,r\}$ rather than $\{0,1,\dotsc,r-1\}$.

\vspace{1em}
From the denominators $d_1$ and $d_2$ of the irreducible fractions of $s_1/r$ and $s_2/r$, 
we can calculate $r$ by $\mathsf{lcm}(d_1, d_2)$.
\begin{align*}
\Pr(\mathsf{lcm}(d_1, d_2) \ne r) &= \Pr_{s_1,s_2\in\{0,\dotsc,r-1\}}(\mathsf{gcd}(s_1,s_2,r) \ne 1)\\
&= \Pr_{s_1,s_2\in\{1,\dotsc,r\}}(\mathsf{gcd}(s_1,s_2,r) \ne 1)\\
&\le \Pr_{s_1,s_2\in\{1,\dotsc,r\}}(\mathsf{gcd}(s_1,s_2) \ne 1)\\
&\le \sum_{p: \text{ prime}} \Pr_{s_1,s_2\in\{1,\dotsc,r\}}(p\mid s_1, p\mid s_2)\\
&\le \sum_{p: \text{ prime}} \frac1{p^2}
\le 0.4523
\end{align*}
\end{frame}

\if0
\begin{frame}{Probability of the best approximation}
\small
Assume that the eigenvalue is $\mathsf{e}^{2\pi i\phi}$.
Let $\varphi\in\{0,1,\dotsc,2^n-1\}$ be the best $n$-bit approximation of $\phi$, i.e., \emm{$|\phi - \frac{\varphi}{2^n}|\le \frac1{2^{n+1}}$}.
\begin{align*}
\Pr(\varphi) &= \frac1{2^{2n}}\left|\left(\sum_{x=0}^{2^n-1} \mathsf{e}^{-i\frac{2\pi}{2^n}\varphi x}\bra{x}\right)\left(\sum_{x=0}^{2^n-1} \mathsf{e}^{i 2\pi\phi x}\ket{x}\right)\right|^2\\
&= \frac1{2^{2n}}\left|\sum_{x=0}^{2^n-1} \mathsf{e}^{2\pi i\left(\phi - \frac{\varphi}{2^n}\right) x}\right|^2\\
&= \frac1{2^{2n}}\left|\frac{1-\mathsf{e}^{2\pi i\left(2^n\phi - \varphi\right)}}{1-\mathsf{e}^{2\pi i\left(\phi - \frac{\varphi}{2^n}\right)}}\right|^2\\
&= \frac1{2^{2n}}\frac{2\sin^2\left(\pi\left(2^n\phi - \varphi\right)\right)}{2\sin^2\left(\pi\left(\phi - \frac{\varphi}{2^n}\right)\right)}\\
&\ge \frac1{2^{2n}}\frac{\sin^2\left(\pi\left(2^n\phi - \varphi\right)\right)}{\left(\pi\left(\phi - \frac{\varphi}{2^n}\right)\right)^2}\\
&\ge \frac1{2^{2n}}\frac{\left(2\left(2^n\phi - \varphi\right)\right)^2}{\left(\pi\left(\phi - \frac{\varphi}{2^n}\right)\right)^2}
= \frac{4}{\pi^2}\approx 0.405
\end{align*}
\end{frame}
\fi

\if0
\begin{frame}{Error probability}
\begin{align*}
\Pr\left(\left|\phi - \frac{\varphi}{2^n}\right| > \frac{c}{2^n}\right)
&=\sum_{} \frac1{2^{2n}}\frac{\sin^2\left(\pi\left(2^n\phi - \varphi\right)\right)}{\sin^2\left(\pi\left(\phi - \frac{\varphi}{2^n}\right)\right)}\\
&= \frac1{2^{2n}}\sum_{}\frac{1}{\left(\pi\left(\phi - \frac{\varphi}{2^n}\right)\right)^2}\\
\end{align*}
\end{frame}
\fi



\begin{frame}{Assignments}
\begin{enumerate}
\setlength{\itemsep}{2em}
\item Show \emm{all} eigenvectors and corresponding eigenvalues of $2^{\lceil \log N\rceil}\times 2^{\lceil\log N\rceil}$ matrix $U_a$.
\end{enumerate}
\end{frame}

\end{document}
